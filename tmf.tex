
%% SP 2008/03/01

\documentclass[preprint,5p,twocolumn,11pt,sort&compress]{elsarticle}

%% For including figures, graphicx.sty has been loaded in

\usepackage{amssymb}

\usepackage{amsthm}
\setlength{\mathindent}{0pt}
\usepackage{tabularx}

\usepackage{graphicx}
\usepackage{epsfig}
\usepackage{textcomp}
\usepackage{subfigure}
\usepackage{natbib}
\usepackage[colorlinks,linkcolor=red,
            				anchorcolor=blue,
            				citecolor=blue]{hyperref}
\usepackage{color,soul}
%\usepackage{xeCJK}
\usepackage{multirow}


\newcommand{\bfsigma}{{\mbox{\boldmath{$\sigma$}}}}
\newcommand{\bfepsilon}{{\mbox{\boldmath{$\varepsilon$}}}}
\newcommand{\dotbfepsilon}{{\mbox{\boldmath{$\dot\varepsilon$}}}}
\newcommand{\dotbfsigma}{{\mbox{\boldmath{$\dot\sigma$}}}}
\newcommand{\bftau}{{\mbox{\boldmath{$\tau$}}}}
\newcommand{\bfpsi}{{\mbox{\boldmath{$\psi$}}}}
\newcommand{\bfphi}{{\mbox{\boldmath{$\phi$}}}}
\newcommand{\bfalpha}{{\mbox{\boldmath{$\alpha$}}}}
\newcommand{\bfbeta}{{\mbox{\boldmath{$\beta$}}}}
\newcommand{\bfK}{{\bf K}}
\newcommand{\bff}{{\bf f}}
\newcommand{\bfn}{{\bf n}}
\newcommand{\bfm}{{\bf m}}
\newcommand{\bft}{{\bf t}}
\newcommand{\bfu}{{\bf u}}
\newcommand{\bfw}{{\bf w}}
\newcommand{\bfa}{{\bf a}}
\newcommand{\bfb}{{\bf b}}
\newcommand{\bfs}{{\bf s}}
\newcommand{\bfB}{{\bf B}}
\newcommand{\bfD}{{\bf D}}
\newcommand{\bfnabla}{{\mbox{\boldmath{$\nabla$}}}}
\newcommand{\bfDelta}{{\mbox{\boldmath{$\Delta$}}}}
\newcommand{\bfkappa}{{\mbox{\boldmath{$\kappa$}}}}
\newcommand{\bfN}{{\bf N}}
\newcommand{\bfT}{{\bf T}}
\newcommand{\bfG}{{\bf G}}
\newcommand{\bfH}{{\bf H}}
\newcommand{\dd}{{\rm d}}
\newcommand{\marked}[1]{\textcolor{red}{#1}}


%\bibliographystyle{elsarticle-num}

\journal{Materials and Design}

\begin{document}

\sethlcolor{yellow}

\begin{frontmatter}

%% Title, authors and addresses

%% use the tnoteref command within \title for footnotes;
%% use the tnotetext command for theassociated footnote;
%% use the fnref command within \author or \address for footnotes;
%% use the fntext command for theassociated footnote;
%% use the corref command within \author for corresponding author footnotes;
%% use the cortext command for theassociated footnote;
%% use the ead command for the email address,
%% and the form \ead[url] for the home page:
%% \title{Title\tnoteref{label1}}
%% \tnotetext[label1]{}
%% \author{Name\corref{cor1}\fnref{label2}}
%% \ead{email address}
%% \ead[url]{home page}
%% \fntext[label2]{}
%% \cortext[cor1]{}
%% \address{Address\fnref{label3}}
%% \fntext[label3]{}

\title{Cyclic Plasticity Modeling of Multi-Axial Thermo-Mechanical Fatigue Tests with Experimental Verification on  Nickel-Based Superalloy Inconel 718}

%% use optional labels to link authors explicitly to addresses:
%% \author[label1,label2]{}
%% \address[label1]{}
%% \address[label2]{}

\author{Jingyu SUN\fnref{label1}}
\author{Huang YUAN\corref{cor1}\fnref{label1}}

%\address[authorlabel1]{Department of Mechanical Engineering, University of Wuppertal, Germany}
\address[label1]{School of Aerospace Engineering, Tsinghua University, Beijing, China\fnref{label1}}
\cortext[cor1]{Corresponding author.}
\ead{yuan.huang@tsinghua.edu.cn}

\begin{abstract}
Thermo-mechanical  and non-proportional loading affect mechanical behavior of metals and change the constitutive modeling. In the present work, extensive experiments are performed for a popular nickel-based superalloy Inconel 718 under both isothermal and thermo-mechanical loading conditions, to investigate the constitutive behavior and computational modeling. Within the frame of the Ohno-Wang cyclic plasticity  a modified constitutive model has been suggested to meet the experimental observations, such as cyclic hardening/softening, non-proportional hardening, thermal-mechanical phase effect, non-masing effect etc. The suggested model agrees with both isothermal as well as thermo-mechanical experiments reasonably. The implicit integration algorithm of the constitutive model is developed and implemented into the commercial finite element code. Comparison between  experimental results and computations confirms that the model can predict the cyclic plastic behavior precisely under most different temperatures and loading paths.
\end{abstract}

%\include{debut}
\begin{keyword}
% keywords here, in the form: keyword \sep keyword
Cyclic plasticity \sep thermo-mechanical fatigue  \sep nickel-based super alloy  \sep multi-axial loading \sep non-masing effects
% PACS codes here, in the form: \PACS code \sep code
% \PACS
\end{keyword}
\end{frontmatter}

% main text
\section{Introduction}
\noindent Gas turbine components experience cyclic multi-axial thermo-mechanical loadings. The increasing performance requirements to modern gas turbines challenge  the material operating limits. Quantified characterization of deformation and fatigue behavior of materials under realistic conditions becomes increasingly of importance for industrial application \cite{harrison1996modelling}, including realistic variations of structure temperature, mechanical loads as well as environmental conditions et al. Inconel 718 is the most popular nickel-based superalloy in turbine industry. It is an oxidation- and corrosion-resistant material possessing optimal thermal and mechanical property. It is used for both rotors and casings, so that the low cycle fatigue characterization of the material is of great significance for mechanical design. The operating temperature of Inconel 718 is limited to 650 $^{\circ}$C for the life limited parts, such as discs and shafts.
A turbine disc is to suffer from high temperature low cycle fatigue (LCF) as well as thermo-mechanical fatigue (TMF).
The LCF and TMF behaviors of the nickel-based superalloy are the major concern for the safety and reliability of gas turbine engines.

Predicting fatigue performance of mechanical components needs a reliable constitutive description of the material.  There were numerous studies on low cycle fatigue of the nickel-based superalloy published.
It is known that the cyclic loading mechanical behavior significantly differs from that under the monotonic loading condition. Engineering metals do not possess stable mechanical behavior until sufficient loading cycles. In the past decades numerous constitutive models for cyclic inelasticity \cite{ohno1993kinematic, Pun2014138, AbdelKarim2000225, Kang2004299} have been published, based on the concept introduced by Armstrong and Frederick \cite{armstrong1966mathematical}. More details can be found in works by Chaboche  \cite{Chaboche20081642} and Abdel-Karim  \cite{AbdelKarim2010711}.

Inconel 718 exhibited a pronounced initial cyclic hardening then became continuous cyclic softening at high strain amplitudes, as reported by  Ye et al. (2004) \cite{ye2004low} . Such experiments imply that the conventional constitutive equations are not suitable to give correct stress and strain predictions in mechanical design. The conventional constitutive models have to be improved to take the multi-axial, thermo-mechanical and cyclic loading conditions into account.
Kobayashi et al. \cite{Kobayashi2008389} and Pereira and Lerch  \cite{Pereira2001715} studied  Inconel 718 behavior using the Johnson-Cook constitutive relation and, however, did not take the cyclic softening behavior into account.
Bai and Wierzbicki  \cite{Bai20081071} modified the classical metal plasticity by taking the dependency of the Lode angle and stress triaxiality into account, which was extended by Algarni et al.  \cite{Algarni2015140} to describe the evolution of yield surface under monotonic loading conditions. Becker and Hackenberg \cite{Becker2011596} suggested a limit surface concept and described the material behavior for a wide temperature range under monotonic and cyclic loading.

Christ and Bauer \cite{Christ201259} investigated the thermo-mechanical behavior of the austenitic stainless steel AISI304L and the near-$\gamma$ TiAl alloy, the TiAl alloy displays a ductile-to-brittle transition in the experimental temperature range.
Farrahi  \cite{Farrahi2014245} applied two plasticity approaches including the spring-slider rule of Nagode and the hardening rule of Chaboche to simulate cyclic behaviors.
The material constants are determined based on the experimental results of low cycle fatigue at various temperatures and
the numerical results demonstrated an effective agreement with tests.
Ohmenh\"{a}user (2014) \cite{Ohmenhauser2014631} adapted a two-layer rheological constitutive model for cyclic thermo-mechanical loading conditions.
The constitutive model parameters is identified as temperature dependent for the austenitic cast steel. The experimental works, however, were not prepared for plasticity modeling. In summary, neither non-proportionality nor multi-axiality of the applied loads was considered under thermo-mechanical cyclic loading conditions.

Development of a reliable cyclic plasticity with experimental verification is necessary for predicting thermo-mechanical fatigue performance of turbine components.
The present work focuses on the  multi-axial cyclic behavior and the stress relaxations of Inconel 718 at varying temperatures between 300 and 650$^{\circ}$C. Effects of non-proportional loading as well as thermo-mechanical coupling are studied. Inconel 718 was proven to be very little influenced by the strain rate below 650$^{\circ}$C \cite{kim1988elevated}, and so the constitutive model is assume to be rate-independent, which is suitable for fatigue life assessment of aero engines. Based on extensive experiments a cyclic plasticity model is suggested for both isothermal and thermo-mechanical loading cases.

\section{Specimens and Experiments}

\subsection{Specimen specification}
\noindent The nickel-based superalloy Inconel 718 investigated in the present work is manufactured by ThyssenKrupp, Germany, and provided in rods of 20mm diameter. The rods were solution-treated following the standard ASTM B637 \cite{ASTMB63716}, to remove initial effects.
The chemical composition of the nickel-based superalloy Inconel 718 in the research is given in Table \ref{Tab:ChemicalCompositionofIN718}. The average grain size is about 15$\rm{\mu m}$.

\begin{table*}[htbp]
  \centering
  \caption{Chemical composition of the investigated material Inconel 718 (wt. \%).}
    \begin{tabular}{llllllllll}
    \hline
    C     & S     & Cr    & Ni    & Mn    & Si    & Mo    & Ti    & Nb    & Cu \\
    \hline
    0.02  & $<$0.001 & 18.53 & 53.44 & 0.05  & 0.06  & 3.06  & 0.99  & 5.30  & 0.04 \\
    \hline
    Fe    & P     & Al    & Pb    & Co    & B     & Ta    & Se    & Bi    &  \\
    \hline
    17.71 & 0.007 & 0.56  & 0.0002 & 0.13  & 0.004 & $<$0.01 & $<$0.0003 & $<$0.00003 &  \\
    \hline
    \end{tabular}%
  \label{Tab:ChemicalCompositionofIN718}%
\end{table*}%


\begin{figure}[!htp]
\centering{\includegraphics[width=8.5cm]{Chapter3Figs/IN718_Axial_Specimen.pdf}}
\centering{\includegraphics[width=8.5cm]{Chapter3Figs/IN718_Multiaxial_Specimen.pdf}}
\caption{Geometry of the specimens investigated in the present work. (a) The solid cylindrical tensile bar. (b) The tubular specimen for tension-torsion tests.}
\label{Fig:Specimen}
\end{figure}


Specimens are illustrated in Figure \ref{Fig:Specimen}, with the surface roughness  $R_a=0.1\rm{\mu m}$. All specimens were manufactured by a CNC machining center and the surfaces were polished after turning. The solid specimens were used for uniaxial tests, whereas the multi-axial fatigue tests were performed in the tubular specimens. The strain gauge length was 12mm and 25mm for solid specimen and tubular specimens, respectively. The uniform temperature in the gage length of the specimens is generated by induction heating. The dimensions of specimens are in accordance with the standards ASTM E606 \cite{astm2012606} and E2207 \cite{standard2007e2207}, respectively. Different multi-axial testing techniques were discussed in \cite{socie2000multiaxial}.


Experiments were performed in the MTS servo hydraulic test system (MTS 809) equipped with  an electric resistance furnace and an induction heating system. Calibration tests confirmed that both heat systems generated comparable testing results, so that effects of heating systems can be neglected. The system is capable of applying axial and biaxial loads with temperature controlled simultaneously.
The shape of the induction coil influences the temperature distribution of the specimen and has to be optimized. In the present work the maximum temperature deviation was $\pm5^{\circ}$C along the 12mm gauge length  and $\pm8^{\circ}$C along the 25mm gauge length.



\subsection{Isothermal tests}
\noindent
The isothermal monotonic tensile tests were performed at four temperatures 20$^{\circ}$C, 300$^{\circ}$C, 550$^{\circ}$C and 650$^{\circ}$C, respectively, heated by the electric resistance furnace. For the duration of each test, the temperature deviations  were less than $\pm3^{\circ}$C. The tensile tests were performed under strain-control with the predetermined strain rate $1.0\times 10^{-4}\rm{s}^{-1}$.
The material properties of different temperatures are summarized in Table \ref{tab:General_material_mechanical_properties}, which shows significant effects from temperature.

\begin{table*}[htbp]
  \centering
  \caption{Results of isothermal tensile tests.}
    \begin{tabular}{ccccc}
    \hline
    Temperature         & Young's modulus   & Yield stress            & Ultimate stress     & Fracture elongation\\
    $T$ [$^{\circ}$C]   & $E$ [GPa]         & $\sigma_{02}$ [MPa]  & $\sigma_u$ [MPa]    & $\varepsilon_f$ [\%]\\
    \hline
    20    & 206.3 & 1192 & 1433 & 26.4 \\
    300   & 190.2 & 1140 & 1334 & 21.2 \\
    550   & 180.2 & 1090 & 1282 & 21.5 \\
    650   & 171.6 & 1064 & 1255 & 23.4 \\
    \hline \\
    \end{tabular}%
  \label{tab:General_material_mechanical_properties}%
\end{table*}%

Both uniaxial and multi-axial strain controlled isothermal low cycle fatigue (LCF) tests were performed with the inductive heating device. The temperatures were measured and controlled by thermo-couples wrapped around the specimen surface at the center of the gauge section of the specimen.
The uniaxial fatigue tests including tension-compression tests and pure torsion tests were performed on solid and tubular specimens respectively. For the duration of each test, the temperature deviations between the indicated and nominal values were less than $\pm3^{\circ}$C.

%\subsubsection{Uniaxial fatigue tests}
%\noindent
Isothermal uniaxial tests were carried out with a given loading ratio $R_{\varepsilon}=\varepsilon_{\min}/\varepsilon_{\max}=-1$.
All uniaxial fatigue tests were performed under strain control.
A triangular waveform was applied at a prescribed frequency of 0.05Hz over the strain range of $\pm 0.6\%$ to $\pm 1.0\%$, i.e. the strain rates were below $1.0\times 10^{-3}\rm{s}^{-1}$ for all tests.
The strain peak  $\varepsilon_{\max}$ and valley $\varepsilon_{\min}$ were given to the controlling software.
The default peak and valley compensation method was used to obtain an accurate feedback value.
All uniaxial fatigue tests were performed at the temperature range of 300$^{\circ}$C to 650$^{\circ}$C in total strain control by the MTS Multipurpose software (MPE).
Failure was deemed to have occurred if the peak load dropped 15\% from the stabilized peak. a 20\% load drop In comparing with the maximum load  was defined as failure, corresponding to nucleation of macro cracks.
The cyclic tests were quitted at 50,000 loading cycles.

\begin{figure}
  \begin{minipage}[t]{0.5\linewidth}
  \nonumber
    \centering
    \includegraphics[width=4.5cm]{Chapter3Figs/LoadPath1.pdf}
    \centerline{\small (a) Tension-Compression.}
  \end{minipage}%
  \begin{minipage}[t]{0.5\linewidth}
    \centering
    \includegraphics[width=4.5cm]{Chapter3Figs/LoadPath2.pdf}
    \centerline{\small (b) Proportional Path.}
  \end{minipage}
  \begin{minipage}[t]{0.5\linewidth}
  \nonumber
    \centering
    \includegraphics[width=4.5cm]{Chapter3Figs/LoadPath3.pdf}
    \centerline{\small (c) Diamond Path.}
  \end{minipage}%
  \begin{minipage}[t]{0.5\linewidth}
    \centering
    \includegraphics[width=4.5cm]{Chapter3Figs/LoadPath4.pdf}
    \centerline{\small (d) Cross Path.}
  \end{minipage}
  \caption{Strain paths in isothermal tests.}
  \label{Fig:LoadPath}
\end{figure}


%\subsubsection{Multi-axial fatigue tests}
%\noindent
The multi-axial cyclic tests involved simultaneous axial and torsional loading with the prescribed phase relationship.
The test procedure was adapt to the standard ASTM E2207-15 which deals with strain-controlled, axial, torsional and combined in-phase and out-of-phase axial torsional fatigue testing with thin-walled tubular specimens under isothermal condition.
The axial strain $\varepsilon$ refers to engineering axial strain, defined as $\varepsilon=\Delta L_g/L_g$. The engineering shear strain $\gamma$ is defined as twist of the specimen.

The thermo-mechanical experiments consisted of axial, torsional and non-proportional cyclic tests with temperature range from 300$^{\circ}$C to 650$^{\circ}$C. The loading conditions for each type of tests are summarized in Table \ref{tab:Loading-Conditions}.
Strain path is described by axial strain $\varepsilon$ and shear strain $\gamma$ in a cycle under the  strain-control conditions. The proportional loading path and the non-proportional loading paths are shown in Figures \ref{Fig:LoadPath}(b) to (d), respectively.

\begin{table*}[htbp]
  \centering
  \caption{Temperature and loading conditions of the isothermal test program.}
    \begin{tabular}{lcccc}
    \hline
    Load type & Temperature & Axial strain range  & Shear strain range & Frequency\\
              & $T$ [$^{\circ}$C] & $\varepsilon$ [\%]& $\gamma /\sqrt 3$ [\%] & $f$ [Hz]  \\
    \hline
    Circle path & 300 & $\pm0.60$ & $\pm0.73$ & 0.05 \\
                & 550 & $\pm0.60$ & $\pm0.73$ & 0.05 \\
                & 650 & $\pm0.60$ & $\pm0.73$ & 0.05 \\
                & 650 & $\pm0.60$ & $\pm0.60$ & 0.05 \\
    \hline
    Diamond path & 300 & $\pm1.00$ & $\pm1.00$ & 0.05 \\
                 & 550 & $\pm1.00$ & $\pm1.00$ & 0.05 \\
                 & 650 & $\pm1.00$ & $\pm1.00$ & 0.05 \\
    \hline
    Cross path   & 300 & $\pm1.00$ & $\pm1.00$ & 0.05 \\
                 & 550 & $\pm1.00$ & $\pm1.00$ & 0.05 \\
                 & 650 & $\pm1.00$ & $\pm1.00$ & 0.05 \\
    \hline
    \end{tabular}%
  \label{tab:Loading-Conditions}%
\end{table*}%




\subsection{Thermo-mechanical fatigue tests}
\noindent
Thermo-mechanical fatigue test is performed under both varying mechanical loading and thermal loading.
Changing the combination of both loads is to investigate interactions between thermal and mechanical stresses in the fatigue process and to figure out a calibration method to describe thermo-mechanical fatigue. As known, the real service load is generally not isothermal. Experiments revealed the thermo-mechanical fatigue life can be significantly shorter than the isothermal one.

According to the standard ASTM E2368-10, a thermo-mechanical fatigue cycle requests uniform temperature and strain fields over the specimen gauge section which vary simultaneously and independently under given conditions. The following factors may influence experimental results:
\begin{itemize}
  \item {\em Thermal strain}, $\varepsilon_{th}$, is induced by temperature and assumed to be linear proportional to temperature increment, $\varepsilon_{th}=\alpha \Delta T$.  It is not related to the mechanical stress.
  \item {\em Mechanical strain}, $\varepsilon_{m}$, is resulting from applied load and directly related with the material mechanical deformation, related to the mechanical stress.
  \item {\em Total strain}, $\varepsilon_t$, is measured by the extensometer and denotes the sum of the thermal and mechanical strains, $\varepsilon_t=\varepsilon_m+\varepsilon_{th}$
  \item {\em Strain ratio}, $ R_{\varepsilon}=\varepsilon_{m,\min}/\varepsilon_{m,\max}$, denotes the loading ratio represented by the minimum mechanical strain over the maximum mechanical strain in a loading cycle.
  \item {\em Phase angle} of the mechanical loading and thermal loading, $\varphi$, represents the time difference between mechanical loading peak and thermal one.
\end{itemize}


In the thermo-mechanical tests the specimen temperature varies in a triangular wave form and cycles between 300$^{\circ}$C and 650$^{\circ}$C, with cooling and heating rates of 3.89$^{\circ}$C/s. The mechanical loads vary in the same speed so that the thermal and mechanical loads remain periodic.
The temperature deviations from the given goal value in the gauge length are between -7$^{\circ}$C and +13$^{\circ}$C.
The thermal strain varying with temperature was determined before the TMF tests and used to calculate the mechanical strain.
Three thermo-mechanical loading cases with different phase angles between 0$^{\circ}$ (in-phase IP), 90$^{\circ}$ phase and 180$^{\circ}$ (out-of-phase OP) are investigated in the present work.
Experimental results are discussed together with computational ones.


\section{The cyclic plasticity model}

\subsection{Formulation of the constitutive model}
\noindent
In the framework of the rate independent and initially isotropic plasticity, the strain rate can be additively decomposed into the elastic and plastic  part, as
\begin{equation}
\dotbfepsilon = {\dotbfepsilon}^e + {\dotbfepsilon}^p.
\end{equation}
The elastic strain and total stress are subjected to Hooke's law,
\begin{equation}
{\dot\bfsigma} = {\mathbb{D}^e}:{\dot\bfepsilon^e},
\label{Equ:HookesLaw}
\end{equation}
where ${\mathbb{D}^e}$ is the fourth-order isotropic elastic stiffness tensor. Hereafter, $(:)$ indicates the inner product between two tensors. $(\dot{\quad})$ denotes the differentiation with respect to time.

The evolution of the plastic strain can be determined from the flow rule as
\begin{equation}
{\dotbfepsilon}^p = \dot \lambda \frac{{\partial F}}{{\partial {\bfsigma}}}
\end{equation}
with
\begin{equation}
F = \sqrt {\frac{3}{2}\left( {\bfs - \bfa} \right):\left( {\bfs - \bfa} \right)}  - Y=0
\end{equation}
as the yield function, where $\bfs$ denotes the deviatoric part of the stress ${\bfsigma}$, the deviatoric backstress $\bfa$ describes the center of the yield surface $F$ in the deviatoric stress space, the yield stress $Y$ is the radius of the yield surface and $\dot \lambda$ is the scalar to be determined using the consistency condition $\dot F = 0$.

The backstress is the essential ingredient of the cyclic plasticity, as initially introduced by Chaboche \cite{Chaboche1986149}, as
\begin{equation}
\bfa = \sum\limits_{k = 1}^M {{\bfa^k}},
\end{equation}
where $M$ denotes the number of partial backstresses.

Following suggestions by D\"orring et al. \cite{Doerring2003} and Abdel Karim et al. \cite{AbdelKarim20051303}, Fang \cite{fang2015cyclic} suggested the evolution equation of the backstress can be expressed as
\begin{equation}
\label{Equ:dotak1}
{\dot\bfa^k} = {h^k}\frac{2}{3}{\dotbfepsilon^p} - H\left( {{f^k}} \right){\dot \omega ^k}\frac{{{\bfa^k}}}{{{r^k}}},
\end{equation}
where ${{{h}}^k} $ and ${{{r}}^k} $ are the state variables introduced to describe cyclic plasticity of the material, $r^k$ is a function of accumulated plastic strain, i.e. ${r^k} = {r^k}\left( p \right)$.  Note no summation convention is applied for $k$. ${{{f}}^k} $ defines a series of the critical state surface in the deviatoric stress space,
\begin{equation}
\label{Equ:fk}
{f^k} = \frac{3}{2}{\bfa^k}:{\bfa^k} - {\left( {{r^k}} \right)^2}.
\end{equation}
Above  $H(\cdot)$ denotes the Heaviside step function. According to Equation (\ref{Equ:fk}), the recovery term is non-zero only when ${f^k} = 0$. In the present work  $r^k$ is assumed to be a function of the accumulative plastic strain and ${\dot \omega ^k}$ is derived from the consistent condition ${\dot f^k} = 0$. It follows
\begin{eqnarray}
\label{Equ:dotfk}
{\dot f^k} &=& 3{\bfa^k}:{\dot\bfa^k} - 2{r^k}{\dot r^k} \\ \nonumber
&=& 3{\bfa^k}:\left[ {{h^k}\frac{2}{3}{{\dotbfepsilon}^p} - H\left( {{f^k}} \right){{\dot \omega }^k}\frac{{{\bfa^k}}}{{{r^k}}}} \right] - 2{r^k}{\dot r^k} = 0.
\end{eqnarray}
When $H\left( {{f^k}} \right) = 1$, it results in
\begin{equation}
\label{Equ:dotomegak}
{\dot \omega ^k} = {h^k}\frac{{{\bfa^k}}}{{{r^k}}}:{\dotbfepsilon^p} - {\dot r^k}.
\end{equation}
Substituting ${\dot \omega ^k}$ into Equation (\ref{Equ:dotak1}) follows
\begin{eqnarray}
\label{Equ:dotak2}\nonumber
{\dot\bfa^k} &=& \frac{2}{3}{h^k}{\dotbfepsilon^p} - H\left( {{f^k}} \right){h^k}\left\langle {\frac{{{\bfa^k}}}{{{r^k}}}:{{\dotbfepsilon}^p}} \right\rangle \frac{{{\bfa^k}}}{{{r^k}}} \\
&+& H\left( {{f^k}} \right)\frac{{{\bfa^k}}}{{{r^k}}}{\dot r^k}
\end{eqnarray}
with $<\cdot>$ as the Macauley brackets.

Combining Equation (\ref{Equ:dotak2}) with Armstrong and Frederick model as by Adbel-Karim and Ohno  \cite{AbdelKarim20051303},  ${\bfa^k}$ obeys the following evolution rule, as
\begin{eqnarray}
\label{Equ:dotak3}
{\dot\bfa^k} &=& {r^k}{\zeta ^k}\left[ \frac{2}{3}{{\dotbfepsilon}^p} - {\mu ^k}\frac{\bfa^k}{r^k}\dot p \right.
\\ \nonumber
&-& \left.  H\left( {f^k} \right) \left\langle {\frac{\bfa^k}{r^k}:{\dotbfepsilon}^p} - {\mu ^k}\dot p \right\rangle \frac{\bfa^k}{r^k} \right]
+ H\left( {f^k} \right)\frac{\bfa^k}{r^k}{\dot r^k},
\end{eqnarray}
where ${\zeta ^k}$ is the material constant related to ${h^k} = {r^k}{\zeta ^k}$. ${\mu ^k}$ is the combination parameter with $0 \leqslant {\mu ^k} \leqslant 1$. When ${\dot r^k} = 0$ and ${\mu ^k} = 0$, the modified model here is reduced to the known Ohno and Wang model. For ${\dot r^k} = 0$ and ${\mu ^k} = 1$, the  model above is reduced to the Armstrong and Frederick model. Thus, for large ${\mu ^k}$, the present model predicts more significant ratcheting and cyclic stress relaxation.


As shown by Kang \cite{Kang2004299}, it is convenient to normalize the variable $\bfa_k$ by $r^k$ as
\begin{equation}
\label{Equ:ak1}
{\bfa^k} = {r^k}{\bfb^k}
\end{equation}
with
\begin{equation}
{f^k} = \frac{3}{2}{\bfb^k}:{\bfb^k} - 1 = 0.
\end{equation}
It follows
\begin{equation}
{\dot\bfb^k} = {\zeta ^k}\left( {\frac{2}{3}{\dotbfepsilon^p} - {\bfb^k}{{\dot p}^k}} \right) + \left[ {H\left( {{f^k}} \right) - 1} \right]\frac{{{{\dot r}^k}}}{{{r^k}}}{\bfb^k},
\end{equation}
\begin{equation}
{\dot p^k} = \left[ {{\mu ^k} + H\left( {{f^k}} \right)\left\langle {\sqrt {\frac{3}{2}} {\bfb^k}:\bfn - {\mu ^k}} \right\rangle } \right]\dot p,
\end{equation}
where
\[\bfn = \frac{{\bfs - \bfa}}{{\left\| {\bfs - \bfa} \right\|}}
\]
is the normal direction of the yield surface. Here $\parallel\cdot\parallel$ is the Euclidean norm of a second rank tensor.


\subsection{Cyclic hardening}
Material testing subjected to alternating loading generates characteristic stress-strain curves. Observations on stress-strain hysteresis loops of austenite stainless steel showed that the peak stresses were not monotonic to plastic strain \cite{fang2015cyclic}. The material demonstrates initial cyclic hardening and then cyclic softening. Before the final failure occurs, the material showed secondary hardening. To describe such complex behavior $r^k$ should consist of two parts, $r_0^k$ and $r_{\Delta}^k$, that is,
\begin{equation}
\label{Equ:rk1}
{r^k} = r_0^k + r_\Delta ^k,
\end{equation}
which can be identified from the monotonic tension curve and cyclic tension-compression curve, respectively.
Above $r_\Delta ^k$ is assumed to depend on the equivalent plastic strain $p$,
\begin{equation}
\label{Equ:rdeltak1}
%r_\Delta ^k = r_{\Delta s}^k\left[ {1 - a_1^k{e^{ - b_1^kp}} + a_2^k({e^{ - b_2^kp}} - {e^{ - b_3^kp}})} \right]
r_\Delta ^k = r_{\Delta s}^k\left[ {1 - a_1^k{e^{ - b_1^kp}} - (1-a_1^k){e^{ - b_2^kp}} }\right].
\label{Equ:rdeltak}
\end{equation}
The dependence on plastic strain allows more accurate description of the cyclic hardening as well as softening behavior of the material.

\subsection{Dynamic recovery and plastic strain memorization}
The plastic strain based memory surface proposed by Chaboche \cite{Chaboche1986149} is defined as
\begin{equation}
\label{Equ:g1}
g = \sqrt {\frac{2}{3}\left( {{\bfepsilon^p} - \bfbeta} \right):\left( {{\bfepsilon^p} - \bfbeta} \right)}  - q,
\end{equation}
where $\bfbeta$ and $q$ represent the radius and center of the non-hardening surface, respectively.
Updating of the memory state is only available when the current plastic strain is on the surface (i.e. $g=0$) and the flow direction is outwards of the surface, that is,
\[
\frac{{\partial g}}{{\partial {\bfepsilon^p}}}:d{\bfepsilon^p} > 0.
\]
Above $\bfbeta$ and $q$ respectively obey the following evolution rules,
\begin{equation}
\label{Equ:dotbeta1}
\dot\bfbeta  = \left( {1 - \eta } \right)H\left( g \right)\left\langle {\bfn:{\bfn^*}} \right\rangle \sqrt {\frac{3}{2}} {\bfn^*}\dot p,
\end{equation}
\begin{equation}
\label{Equ:dotq1}
\dot q = \eta H\left( g \right)\left\langle {\bfn:{\bfn^*}} \right\rangle \dot p
\end{equation}
with
\begin{equation}
\label{Equ:nstar}
{\bfn^*} = \frac{{\partial g}}{{\partial {\bfepsilon^p}}} = \frac{{{\bfepsilon^p} - \bfbeta}}{{\left\| {{\bfepsilon^p} - \bfbeta} \right\|}}.
\end{equation}

\subsection{Non-proportional hardening}
In the present work, the non-proportional hardening is assumed as an additional term of the isotropic hardening, as
\begin{equation}
Y = {Y_0} + {Y_{\Delta np}},
\end{equation}
where $Y_0$ denotes the yield stress for proportional loading and ${Y_{\Delta np}}$ is the non-proportional hardening defined by
\begin{equation}
{\dot Y_{\Delta np}} = {\gamma _p}\left( {{Y_{sat}} - {Y_{\Delta np}}} \right)\dot p.
\end{equation}
Integrating the above equation with the initial condition ${Y_{\Delta np}}(0)=0$ follows
\begin{equation}
{Y_{\Delta np}} = Y_{sat}\left( 1-e^{- \gamma_p p} \right),
\end{equation}
where $Y_{sat}$ is the saturated value of ${Y_{\Delta np}}$ under a non-proportional loading path.
Tanaka introduced a non-proportionality factor \cite{tanaka1994nonproportionality},
\begin{equation}
\phi  = \sqrt {1 - \frac{{\bfn:\mathbb{C}:\mathbb{C}:\bfn}}{{\mathbb{C}::\mathbb{C}}}}
\end{equation}
to quantify influence of the non-proportional load path to the material behavior.  $\mathbb{C}$ is a fourth-rank tensor and represents the internal dislocation structure \cite{tanaka1994nonproportionality}, as
\begin{equation}
\dot {\mathbb{C}} = {c_c}\left( {\bfn \otimes \bfn - \mathbb{C}} \right)\dot p.
\end{equation}
Assuming ${Y_{sat}}$ depending on the loading path follows the simplest form as
\begin{equation}
{Y_{sat}} = \phi {Y_{\Delta NS}},
\end{equation}
with $0 \leq \phi \leq 1$.
For the proportional loading $\phi = 0$, there is no non-proportional hardening, while $\phi = 1$ means ${Y_{sat}} = {Y_{\Delta NS}}$ for the maximum non-proportional hardening.

As discussed in \cite{fang2015cyclic}, the evolution of ${Y_{\Delta NS}}$ can be defined as
\begin{equation}
{\dot Y_{\Delta NS}} = {\gamma _q}\left( {{Y_{\Delta 0}} - {Y_{\Delta NS}}} \right)\dot q.
\end{equation}
Noting that the strain memory effect is taken into account, $q$ is the radius of strain memory surface, ${\gamma _q}$ is a rate parameter.
If the material shows Massing behavior,  ${\gamma _q}=0$. $Y_{\Delta 0} $ denotes the saturate value of $Y_{\Delta NS}$.

\section{Integration algorithm of the constitutive equations}
\label{}
The implicit algorithm of the constitutive equation should provide the stress increment and the plastic strain increment, based on the given strain increment.
The backward Euler method is a first order implicit method popularly used in the finite element method. Considering the integration interval from Step $n$ to Step $n+1$, variables of Step $n$ are denoted by the index $n$. The symbol $\Delta$ indicates the increment of the variable from $n$ to $n+1$. It is assumed that the strain increment can be decomposed in the current configuration, as
\begin{equation}
{\Delta\bfepsilon} = \Delta\bfepsilon^e + \Delta\bfepsilon^p.
\end{equation}
The stress increment can be determined from the Hooke's law as
\begin{equation}
\label{Equ:sigman+1}
\Delta\bfsigma= {\mathbb{D}^e}:\left( {{\Delta\bfepsilon} - \Delta\bfepsilon^p} \right),
\end{equation}
where $\mathbb{D}^e$ represents the elasticity matrix, ${\mathbb{D}^e} = \kappa {\mathbf{I}} \otimes {\mathbf{I}} + 2G\mathbb{I}$ for elastically isotropic materials. Here $G$ is shear modulus and  $\kappa  = K - \frac{2}{3}G$  is Lame constant. ${\mathbf{I}}$ and $\mathbb{I}$ respectively indicate the identity tensors of the second- and the fourth-order, respectively, given in the component form as ${I_{ij}} = {\delta _{ij}}$ and ${\mathbb{I}_{ijkl}} = {\delta _{ik}}{\delta _{jl}}$ with ${\delta _{ij}}$ as Kronecker delta.
The return mapping algorithm is popular in computational plasticity, which consists of an elastic predictor and a plastic corrector. The elastic predictor is taken as the elastic tentative stress,
\begin{equation}
{\bfsigma}^{tr} = {\mathbb{D}^e}:\left( {{\bfepsilon} - \bfepsilon_n^p} \right).
\end{equation}
The relationship between the stress at Step $n+1$  and the trial stress is given through
\begin{equation}
\label{Equ:sigma1}
{{\bfsigma}^{} = {\mathbb{D}^e}:\left( {\bfepsilon^{} - \bfepsilon^p} \right)} = {\bfsigma}^{tr} - 2G\Delta \bfepsilon^p.
\end{equation}
Note that, because of elastic isotropy and plastic incompressibility, $\Delta \bfepsilon^p$ is a deviatoric tensor with ${\rm Tr}\left( {\Delta \bfepsilon^p} \right)=0$.
The stress deviator is given by
\begin{equation}
\bfs \equiv {\bfsigma} - \frac{1}{3}\left( {{\bfsigma}:{\mathbf{I}}} \right){\mathbf{I}} = {\mathbb{I}_d}:{\bfsigma},
\end{equation}
where ${\mathbb{I}_d} = \mathbb{I} - \frac{1}{3}{\mathbf{I}} \otimes {\mathbf{I}}$ is the deviatoric projection tensor.
Taking the deviatoric part of Equation (\ref{Equ:sigma1}) follows
\begin{equation}\label{Equ:sminusa1}
{\bfs} = \bfs^{tr} - 2G\Delta \bfepsilon^p.
\end{equation}

\begin{figure}[!htp]
\centering{\includegraphics[width=9cm]{Chapter4Figs/radial_return_map.pdf}}
\caption{Schematics of the two steps radial return map method.}
\label{}
\end{figure}

A two steps radial return map method proposed by  Kobayashi and Ohno \cite{kobayashi2002implementation} are employed for integrating the present constitutive model. The backstress is written as
\begin{equation}
{\bfa} = \sum\limits_{k = 1}^M {r^k\bfb^k}.
\end{equation}
Then the tentative form of the backstress is determined from
\begin{equation}
\label{Equ:btrn+1}
\bfb^{k,tr} = \bfb_n^k + \frac{2}{3}{\zeta ^k}\Delta \bfepsilon^p.
\end{equation}
Therefore, the normalized backstress $\bfb^k$ can be written as
\begin{equation}
\label{Equ:bn+1}
\bfb^k = \theta ^k\left( {\bfb_n^k + \frac{2}{3}{\zeta ^k}\Delta \bfepsilon^p} \right) = \theta ^k\bfb^{k,tr},
\end{equation}
where $\theta ^k$ is defined by
\begin{equation}
\label{Equ:thetan+1}
\theta ^k = c^k + H\left( {\bar f^k} \right)\left( {\frac{1}{{\bar b^{k,tr}}} - c^k} \right)
\end{equation}
 and satisfies $0 < \theta ^k \leqslant 1$. Above $c^k$ and ${\bar f^k}$ are given through
\[
c^k = \frac{1}{{1 + {\zeta ^k}\mu \Delta {p} - \frac{{\Delta r^k}}{{r^k}}}},
\]
\[
\bar f^k = {\left( {c^k\bar b^{k,tr}} \right)^2} - 1,
\]
with
\[
\bar b^{k,tr} = \sqrt {\frac{3}{2}\bfb^{k,tr}:\bfb^{k,tr}}.
\]
Substituting ${\bfa}$ into Equation (\ref{Equ:sminusa1}) follows
\begin{eqnarray}
\label{Equ:sminusa2}
%\begin{aligned}
{{\bfs} - {\bfa}}&=&
\bfs^{tr} - \sum\limits_{k = 1}^M {r^k\theta ^k\bfb_n^k}
\\ \nonumber
&-& \left( {2G + \frac{2}{3}\sum\limits_{k = 1}^M {r^k\theta ^k{\zeta ^k}} } \right)\Delta \bfepsilon^p.
%\end{aligned}
\end{eqnarray}
From the yield function and the flow rule one obtains finally
\begin{equation}
\Delta {p} = \frac{{\sqrt {\frac{3}{2}} \left\| {\bfs^{tr} - \sum\limits_{k = 1}^M {r^k\theta ^k\bfb_n^k} } \right\| - {Y_n}}}{{{{\left( {\frac{{\partial Y}}{{\partial p}}} \right)}} + 3G + \sum\limits_{k = 1}^M {r^k\theta ^k{\zeta ^k}} }}
\end{equation}
and
\begin{equation}
{\bfs} - {\bfa} = \frac{{{Y}\left( {\bfs^{tr} - \sum\limits_{k = 1}^M {r^k\theta ^k\bfb_n^k} } \right)}}{{\sqrt {\frac{3}{2}} \left\| {\bfs^{tr} - \sum\limits_{k = 1}^M {r^k\theta ^k\bfb_n^k} } \right\|}}.
\end{equation}



\section{Parameter identification}
The model parameters can be divided into the following categories: Elastic-plastic parameters, isotropic hardening parameters, kinematic hardening parameters and cyclic hardening/softening parameters. The parameters for non-proportional loading hardening is included in the isotropic hardening.

\subsection{Determination of elastic-plastic constants}

The elastic parameters were determined from the tensile tests directly. For the elastically isotropic material Hooke's law can be written as
\begin{eqnarray}
   \bfepsilon = \frac{1}{E}(\bfsigma - \nu[\mathrm{Tr}(\bfsigma)~\mathbf{I} - \bfsigma]),
\label{Equ:HookesLawInEandNu}
\end{eqnarray}
where $E$ is  Young's modulus and $\nu$ is Poisson's ratio, depending on temperature.

By neglecting the plastic strain amplitude effect, ${\zeta ^k}$ and  $r_0^k$ can be determined using the following equations \cite{Jiang1996387, jiang1996modeling}, if no isotropic hardening is assumed,
\begin{equation}
{\zeta ^k} = \frac{1}{{\varepsilon _p^k}},
\label{Eqn:zetak}
\end{equation}
\begin{equation}
r_0^k = \left( {\frac{{\sigma _{}^k - \sigma _{}^{k - 1}}}{{\varepsilon _p^k - \varepsilon _p^{k - 1}}} - \frac{{\sigma _{}^{k + 1} - \sigma _{}^k}}{{\varepsilon _p^{k + 1} - \varepsilon _p^k}}} \right)\varepsilon _p^k,
\end{equation}
where $\sigma^k$ and $\varepsilon_p^k( k = 0,1,2,...,M+1 )$ denote the stress and plastic strain at the $k$th point on the monotonic  stress-plastic strain curve, respectively. $\sigma^0$ is the initial yield stress related to the zero plastic strain $\varepsilon_p^0=0$. Generally, the uniaxial tensile stress-strain curve is represented by $M$ points.

In the constitutive model, $r^k$ is the saturated value and has to be determined from cyclic stress-strain curve.
The cyclic stress-strain curve can be obtained form the experimental stabilized hysteresis loop.
The half life ($N_f/2$)th. cycle is assumed to be the stabilized hysteresis loop.

By combining Equations (\ref{Equ:rk1}) and (\ref{Equ:rdeltak1}), $r^k$ can be expressed as:
\begin{equation}\label{rk}
{r^k} = r_0^k + r_{\Delta s}^k\left[ {1 - a_1^k{e^{ - b_1^kp}} - (1-a_1^k){e^{ - b_2^kp}} }\right],
\end{equation}
where $r_{\Delta s}^k$, $a_1^k$, $b_1^k$ and $b_2^k$ are  constants and $p$ is the equivalent plastic strain.
The saturated value of $r^k$ is denoted as $r^k_s$ and the saturated increment value of $r^k$ is denoted as $r^k_{\Delta s}$.
A simple procedure for determining the material constants $r_{\Delta s}^k$ was by assuming that $p$ is infinitely large,
\begin{equation}
r_s^k=\mathop {\lim }\limits_{{\rm{p}} \to \infty } {r^k} = r_0^k + r_{\Delta s}^k.
\label{Equ:rsk}
\end{equation}
Generally, when the cyclic hardening/softening is stabilized, the equivalent plastic strain is large.
Consequently, the data in the stabilized hysteresis loop can be used to determine the saturated values $r^k_s$ and $r^k_{\Delta s}$.

The calculated alternating stress-strain curve was  divided into $M=9$ segments with the same plastic strain for both monotonic and cyclic stress-strain curve. The parameters $\zeta^k$ and $r_0^k$ can be determined from monotonic tension test as suggested by Jiang and Kurath \cite{Jiang1996387}. From the calculated alternating stress-strain curve the same procedure for $\zeta^k$ and $r_0^k$ are used to determine the saturated values $r^k_s$ and $r^k_{\Delta s}$.  $r^k$ is determined from the uniaxial stable hysteresis loop curve following Equation (\ref{rk}). ${\zeta ^k}$ for the cyclic stress-strain relation are the same as those for the monotonic case.

%\begin{figure}[!htp]
%\centering{\includegraphics[width=8.5cm]{Chapter6Figs/Determine_rsk.pdf}}
%\caption{IN718 uniaxial monotonic tensile and calculated alternating stress versus plastic strain curves used to determine the material constants $r_s^k$ and $r_{\Delta s}^k$ at 650$^{\circ}$C.}
%\label{Fig:Determine_rsk}
%\end{figure}

As discussed above, both monotonic tension curve and the stabilized alternating stress-strain curve are used for identifying the model parameters.
$r_{\Delta s}^k$ is describing the deviation magnitude of the two curves and the parameters $a_{1}^k$, $b_{1}^k$ and $a_{2}^k$ are used to describe the softening rate of the material.
$a_1^k$, $b_1^k$, $a_2^k$, $b_2^k$, $b_3^k$ are determined from the history curve of the peak stress under uniaxial cyclic loading. For simplicity, one assumes that the cyclic softening is purely kinematic under uniaxial loading conditions, as introduced by Ohno et al \cite{Ohno1993375}.
$a_{1}^k$, $b_{1}^k$ and $b_{2}^k$ are independent of $k$, i.e. all segments share the same parameters, $a_1, b_2$ and $b_2$.
Thus, the parameters can be determined from the cyclic stress-strain curve,
\begin{equation}
1 - \frac{{\sigma  - {\sigma _0}}}{{{\sigma _0} - {\sigma _{\min }}}} = 1 - a_1{e^{ - b_1 p}} - \left( {1 - a_1} \right){e^{ - b_2 p}}.
\end{equation}
The relation provides optimal agreement between experimental curve and model, as shown in Fig.  \ref{Fig:Fitting300C_7041_Plot}. Fang \cite{fang2015cyclic} proposed a simplified model as
\begin{equation}
r_\Delta ^k = r_{\Delta s}^k\left( 1 - a_1^k{e^{ - b_1^kp}}\right).
\label{Equ:fangrdeltak}
\end{equation}
One observes significant deviations at small plastic strain region.

$Y_{\Delta NS}$, $\gamma_p$ and $\gamma_q$ are fitted from the stabilized non-proportional cyclic stress-strain curve;
parameters $\mu$ and $c_c$ are suggested as the values proposed by Chaboche \cite{Chaboche1986149} and Tanaka \cite{tanaka1994nonproportionality}.

%\begin{figure}[!htp]
%\centering{\includegraphics[width=8.5cm]{Chapter6Figs/EpVsPeakStress300C_7041.pdf}}
%\caption{Axial peak stress versus accumulated plastic strain at 300$^{\circ}$C.}
%\label{Fig:EpVsPeakStress300C_7041}
%\end{figure}

\begin{figure}[!htp]
\centering{\includegraphics[width=8.5cm]{Chapter6Figs/Fitting300C_7041_Plot.pdf}}
\caption{Kinematic hardening from the full reversed strain cycling tests with comparison to different hardening models of $r^k$.}
\label{Fig:Fitting300C_7041_Plot}
\end{figure}



\subsection{Temperature-dependent parameters}
In this section, the model parameters are determined under isothermal conditions.
According to isothermal uniaxial monotonic tension tests and  cyclic stress-strain hysteresis curves at different temperatures, the material constants at 300$^{\circ}$C, 550$^{\circ}$C and 650$^{\circ}$C are identified and summarized in Table \ref{tab:General_material_mechanical_properties} and Table \ref{tab:3} with $M=9$. The temperature dependence of the parameters  are linearly interpolated in the intervals.
%The identification of the material parameters $\zeta^k$, $r_0^k$, $r_{\Delta s}^k$ $E$, $\mu$, $\sigma_{p=0.2}$


%\begin{figure}
%  \begin{minipage}[t]{0.5\linewidth}
%  \nonumber
%    \centering
%    \includegraphics[width=3in]{eps/Monotonic_tension.eps}
%    \centerline{(a) Tension test.}
%    \label{Fig:side:a}
%  \end{minipage}%
%  \begin{minipage}[t]{0.5\linewidth}
%    \centering
%    \includegraphics[width=3in]{eps/E.eps}
%    \centerline{(b) Young's modulus.}
%    \label{Fig:side:b}
%  \end{minipage}
%  \caption{Monotonic tension test and Young's modulus at different temperatures.}
%  \label{Fig:Monotonic_tension}
%\end{figure}

%\begin{figure}[!htp]
%\centering{\includegraphics[width=8.5cm]{Chapter6Figs/eps/E.eps}}
%\caption{Young's modulus at different temperature.}
%\label{Fig:E}
%\end{figure}


%\begin{table*}[htbp]
%  \centering
%  \caption{Material properties of the proposed constitutive model.}
%    \begin{tabular}{rr|rr|rr|rr}
%    \hline
%          &       & \multicolumn{2}{c}{300$^{\circ}$C} & \multicolumn{2}{c}{550$^{\circ}$C} & \multicolumn{2}{c}{650$^{\circ}$C} \\
%    \hline
%    $k$   & $\zeta^k$ & $r_0^k$ & $r_{\Delta s}^k$ & $r_0^k$ & $r_{\Delta s}^k$ & $r_0^k$ & $r_{\Delta s}^k$ \\
%    1     & 20000 & 292.65  & -37.83  & 242.66  & -100.14  & 227.08  & -139.00  \\
%    2     & 10000 & 60.11  & -10.89  & 57.06  & -13.56  & 54.94  & -15.16  \\
%    3     & 5000  & 74.70  & -14.30  & 71.17  & -17.29  & 68.47  & -18.98  \\
%    4     & 2000  & 89.45  & -18.06  & 85.54  & -21.23  & 82.22  & -22.89  \\
%    5     & 1000  & 82.86  & -17.20  & 79.54  & -20.15  & 76.38  & -21.36  \\
%    6     & 500   & 91.27  & -20.20  & 87.90  & -22.68  & 84.34  & -23.68  \\
%    7     & 250   & 113.42  & -26.21  & 109.64  & -28.84  & 105.11  & -29.65  \\
%    8     & 100   & 135.81  & -32.72  & 131.78  & -35.34  & 126.21  & -35.76  \\
%    9     & 50    & 140.01  & -32.38  & 136.93  & -33.02  & 130.89  & -32.75  \\
%    \hline
%    \multicolumn{8}{l}{300$^{\circ}$C,$a_1^k$=0.4,$b_1^k$=1.27,$b_2^k$=9.45} \\
%    \multicolumn{8}{l}{550$^{\circ}$C,$a_1^k$=0.41,$b_1^k$=0.87,$b_2^k$=11.8} \\
%    \multicolumn{8}{l}{650$^{\circ}$C,$a_1^k$=0.42,$b_1^k$=0.49,$b_2^k$=9.18} \\
%    \multicolumn{8}{l}{$\mu=0.2$,$\gamma_p$=10,$\gamma_q$=50,$Y_{\Delta NS}$=100,$c_c$=50} \\
%    \hline
%    \end{tabular}%
%  \label{tab:3}%
%\end{table*}%

% Table generated by Excel2LaTeX from sheet 'Sheet2'
\begin{table*}[htbp]
  \centering
  \caption{Material properties (stress: MPa, strain: mm/mm).}
    \begin{tabular}{rrrrrrrrrrr}
    \hline
          & $k$   & 1     & 2     & 3     & 4     & 5     & 6     & 7     & 8     & 9 \\
          & $\zeta^k$ & 20000 & 10000 & 5000  & 2000  & 1000  & 500   & 250   & 100   & 50 \\
    \hline
    300$^{\circ}$C & $r_0^k$ & 292.65  & 60.11  & 74.70  & 89.45  & 82.86  & 91.27  & 113.42  & 135.81  & 140.01  \\
          & $r_{\Delta s}^k$ & -37.83  & -10.89  & -14.30  & -18.06  & -17.20  & -20.20  & -26.21  & -32.72  & -32.38  \\
          & & \multicolumn{9}{l}{$a_1^k$=0.4, $b_1^k$=1.27, $b_2^k$=9.45, $k$=1 to 9} \\
    \hline
    550$^{\circ}$C & $r_0^k$ & 242.66  & 57.06  & 71.17  & 85.54  & 79.54  & 87.90  & 109.64  & 131.78  & 136.93  \\
          & $r_{\Delta s}^k$ & -100.14  & -13.56  & -17.29  & -21.23  & -20.15  & -22.68  & -28.84  & -35.34  & -33.02  \\
          & & \multicolumn{9}{l}{$a_1^k$=0.41, $b_1^k$=0.87, $b_2^k$=11.8, $k$=1 to 9} \\
    \hline
    650$^{\circ}$C & $r_0^k$ & 227.08  & 54.94  & 68.47  & 82.22  & 76.38  & 84.34  & 105.11  & 126.21  & 130.89  \\
          & $r_{\Delta s}^k$ & -139.00  & -15.16  & -18.98  & -22.89  & -21.36  & -23.68  & -29.65  & -35.76  & -32.75  \\
          & & \multicolumn{9}{l}{$a_1^k$=0.42, $b_1^k$=0.49, $b_2^k$=9.18, $k$=1 to 9} \\
    \hline
          & & \multicolumn{9}{l}{$\mu=0.2$,$\gamma_p$=10,$\gamma_q$=50,$Y_{\Delta NSs}$=100,$c_c$=50} \\
    \hline
    \end{tabular}%
  \label{tab:3}%
\end{table*}%

\section{Computational analysis}

\subsection{Simulations of isothermal fatigue tests}
\label{}


\begin{figure*}
\centering{\includegraphics[width=8.5cm]{Chapter6Figs/IN718_Isothermal_1st.pdf}}
\centering{\includegraphics[width=8.5cm]{Chapter6Figs/IN718_Isothermal_200th.pdf}}
\caption{Comparison of the stress-strain hysteresis loops between experiments and computations  under isothermal uniaxial tensile loading conditions at 300$^{\circ}$C, 550$^{\circ}$C and 650$^{\circ}$C, respectively. (a) The first loading cycle. (b) The 200th cycle.}
\label{Fig:200th_Exp_Sim}
\end{figure*}

The uniaxial tension and torsion test can be modeled using a single axisymmetric tension-torsion element.
Details of the ABAQUS use-defined material model UMAT has been reported in the previous sections.
The stress-strain hysteresis loops of the uniaxial tests are simulated and predicted by the proposed model under the symmetrical strain path.
Comparison between  computational stress-strain and  experimental loops is shown in Fig. \ref{Fig:200th_Exp_Sim}.
The figure illustrates results for first loading cycles and the 200th cycles at 300$^{\circ}$C, 550$^{\circ}$C and 650$^{\circ}$C, respectively.

The first strain controlled loop consists of initial loading, unloading and reverse loading.
As shown in Fig. \ref{Fig:200th_Exp_Sim}(a), in the initial loading cycle the predicted and experimental loops are in reasonable agreement, but deviations are observed during reverse plastic loading for the lower temperatures because the kinematic hardening variable $r^k$ approximates ${r_0^k}$ in the first loading loop.
Note that the kinematic hardening variable ${r_0^k}$ is obtained from monotonic tension tests and evolution of $r^k$ depends on the accumulated plastic strain $p$.
During the first reverse plastic loading, the accumulated plastic strain $p$ is small so the first simulated stress-strain hysteresis loops are mainly based on the monotonic curve. It leads to the deviation of the computation from the experimental result in the initial loading cycle.
Along loading and unloading cycles, with evolution of $r_s^k$ (see Equation (\ref{Equ:rsk})), the predicted responses approaches experimental curve and agreement at the 200th cycle becomes optimal, as shown in \ref{Fig:200th_Exp_Sim}(b). Since the fatigue is determined mainly by the peak and valley values, such deviations will not change fatigue life assessment.


\begin{figure}
\centering{\includegraphics[width=8.5cm]{Chapter6Figs/IN718_Isothermal_Axial+-1_PV_Exp_vs_Sim.pdf}}
\caption{Comparison of experimental and predicted peak and valley stresses under isothermal uniaxial tensile loading conditions at 300$^{\circ}$C, 550$^{\circ}$C and 650$^{\circ}$C, respectively.}
\label{Fig:Compare_PACC-PV_stress_temperature}
\end{figure}


\begin{figure*}[!]
\centering{\includegraphics[width=8.5cm]{Chapter6Figs/Circle_1st.pdf}}
\centering{\includegraphics[width=8.5cm]{Chapter6Figs/Circle_200th.pdf}}
\caption{Comparison between experiments and computations  under isothermal non-proportional loading conditions at  300, 550, 650$^{\circ}$C. (a) The first loading cycle. (b) The 200th cycle.}
\label{Fig:Circle_Exp_Sim}
\end{figure*}


More details from the experiments and computations reveal that the material is cyclic softening, as shown in Fig. \ref{Fig:Compare_PACC-PV_stress_temperature}. The softening behavior becomes more severe as  temperature increases.
As mentioned earlier, in the present constitutive relation, cyclic softening is modeled by the kinematic hardening variable $r^k$, with its saturation value $r_s^k$ smaller than its initial value $r_0$.
The figure confirms a good agreement between experiments and the model in the whole loading and unloading processes.


The multi-axial non-proportional loading results are summarized in Fig. \ref{Fig:Circle_Exp_Sim}, for the circular tension-torsion loading cases at 300$^{\circ}$C, 550$^{\circ}$C and 650$^{\circ}$C, respectively. The cyclic deformation curves were constructed by plotting the amplitudes from peak tensile to peak compressive stress for the first cycle and the 200th cycle with total strain range $\pm1\%$, respectively. At 650$^{\circ}$C the computational prediction shows slight over-estimate near the maximum normal stress, while others reveal optimal agreement for both initial and cyclic loading.

Figure \ref{Fig:IN718_Isothermal_Axial+-1_PV_Exp_vs_Sim_1}(a) depicts the peak stress from the circular loading at the different temperatures.
The alloy exhibits an obvious continuous cyclic softening at higher temperature, while the cyclic softening may reach a saturation stage at low temperature. Such observation can be critical for mechanical components of high temperature.
Fig. \ref{Fig:IN718_Isothermal_Axial+-1_PV_Exp_vs_Sim_1}(a) illustrates effects of kinematic hardening variables. While the Fang's model  \cite{fang2015cyclic}  in Equation (\ref{Equ:fangrdeltak}) shows a clear disagreement to the experiments in Fig. \ref{Fig:IN718_Isothermal_Axial+-1_PV_Exp_vs_Sim_1}(b) , the present prediction in Equation (\ref{Equ:rdeltak}) provides a consistent cyclic behavior in the whole loading history. One term expression of Fang cannot describe the softening behavior properly.

\begin{figure*}[!htp]
\centering{\includegraphics[width=8.5cm]{Chapter6Figs/IN718_Isothermal_Axial+-1_PV_Exp_vs_Sim_2.pdf}}
\centering{\includegraphics[width=8.5cm]{Chapter6Figs/IN718_Isothermal_Axial+-1_PV_Exp_vs_Sim_1.pdf}}
\caption{Comparison of the peak stress under isothermal circular non-proportional loading conditions at 300$^{\circ}$C, 550$^{\circ}$C and 650$^{\circ}$C, respectively. (a) The presented model. (b) Fang's model  \cite{fang2015cyclic} .}
\label{Fig:IN718_Isothermal_Axial+-1_PV_Exp_vs_Sim_1}
\end{figure*}

To study effects of complex non-proportional loading path, an experiment with loading cross path (Fig. \ref{Fig:LoadPath}(d)) is plotted in Fig. \ref{Fig:IN718_Isothermal_300C_7049_XPath_Exp_vs_Sim}, together with computational predictions. Due to prompt changes in loading direction, the stress-strain curves become non-symmetric. In the figure both computational results with and without non-proportional hardening term are present. The obvious disagreement between computations and experiments is observed at the peaks. However, the improvement of the non-proportional hardening in the model is significant, as shown in Fig. \ref{Fig:IN718_Isothermal_300C_7049_XPath_Exp_vs_Sim}. The non-proportional hardening term is meaningful for the constitutive modeling.

\begin{figure*}[!htp]
\centering{\includegraphics[width=8.5cm]{Chapter6Figs/IN718_Isothermal_300C_7049_XPath_Exp_vs_Sim_0.pdf}}
\centering{\includegraphics[width=8.5cm]{Chapter6Figs/IN718_Isothermal_300C_7049_XPath_Exp_vs_Sim_200.pdf}}
\caption{Comparison between experiments at 300$^{\circ}$C under the cross loading path with computational predictions. (a) Without non-proportional hardening term. (b) With the non-proportional hardening factor $Y_{sat}$ = 200MPa.}
\label{Fig:IN718_Isothermal_300C_7049_XPath_Exp_vs_Sim}
\end{figure*}

%\begin{figure*}[!htp]
%\centering{\includegraphics[width=8.5cm]{Chapter6Figs/MisesStressWithoutNPHardening.pdf}}
%\centering{\includegraphics[width=8.5cm]{Chapter6Figs/MisesStressWithNPHardening.pdf}}
%\caption{Comparison between experiment at 300$^{\circ}$C under the cross loading path and computations by using different non-proportional hardening factors. The Mises stress is compared directly. (a) Without non-proportional hardening. (b) Non-proportional hardening factor = 200MPa.}
%\label{Fig:IN718_Isothermal_300C_7049_XPath_Exp_vs_Sim_MisesStress}
%\end{figure*}

%\begin{figure*}
%  \begin{minipage}[t]{0.5\linewidth}
%  \nonumber
%    \centering
%    \includegraphics[width=3.5in]{Chapter6Figs/IN718_Isothermal_300C_7049_XPath_Exp_vs_Sim_0.pdf}
%    \centerline{(a) Without proportional hardening.}
%    \label{Fig:IN718_Isothermal_300C_7049_XPath_Exp_vs_Sim_0}
%  \end{minipage}%
%  \begin{minipage}[t]{0.5\linewidth}
%    \centering
%    \includegraphics[width=3.5in]{Chapter6Figs/IN718_Isothermal_300C_7049_XPath_Exp_vs_Sim_200.pdf}
%    \centerline{(b) Non-proportional hardening factor = 200MPa.}
%    \label{Fig:IN718_Isothermal_300C_7049_XPath_Exp_vs_Sim_200}
%  \end{minipage}
%  \caption{Experiments at 300$^{\circ}$C under the cross loading path, comparing with computations by using different non-proportional hardening factors.}
%  \label{Fig:IN718_Isothermal_300C_7049_XPath_Exp_vs_Sim}
%\end{figure*}

%\begin{figure*}
%  \begin{minipage}[t]{0.5\linewidth}
%  \nonumber
%    \centering
%    \includegraphics[width=3.5in]{Chapter6Figs/MisesStressWithoutNPHardening.pdf}
%    \centerline{(a) Without non-proportional hardening.}
%    \label{Fig:MisesStressWithoutNPHardening}
%  \end{minipage}%
%  \begin{minipage}[t]{0.5\linewidth}
%    \centering
%    \includegraphics[width=3.5in]{Chapter6Figs/MisesStressWithNPHardening.pdf}
%    \centerline{(b) Non-proportional hardening factor = 200MPa.}
%    \label{Fig:MisesStressWithNPHardening}
%  \end{minipage}
%  \caption{Comparison between experiment at 300$^{\circ}$C under the cross loading path and computations by using different non-proportional hardening factors. The Mises stress is compared directly.}
%  \label{Fig:IN718_Isothermal_300C_7049_XPath_Exp_vs_Sim_MisesStress}
%\end{figure*}


\begin{figure*}
\centering{\includegraphics[width=8.5cm]{Chapter6Figs/All_TMF_+-1_IP_10th.pdf}}
\centering{\includegraphics[width=8.5cm]{Chapter6Figs/IN718_TMF_Axial+-1_PV_Exp_vs_Sim_ALL.pdf}}
\caption{Experimental and computational results under the multi-axial thermo-mechanical loading conditions with temperature between 300 and 650$^{\circ}$C. (a) The hysteresis loops in the in-phase loading, out-of-phase loading and the loading with a phase angle of 90$^\circ$. (b) Peak-valley stresses as function of the loading cycles in the in-phase loading, out-of-phase loading and and the loading with a phase angle of 90$^\circ$.}
\label{Fig:TMF_IP}
\end{figure*}


\subsection{Uniaxial thermo-mechanical fatigue tests}
Under thermo-mechanical loading conditions both mechanical loads as well as specimen temperature vary, which cause different deformation and damage mechanisms in the material.
In the present section both experimental and computational results for the varying temperature between 300-650$^{\circ}$C are considered, where tests are controlled by the tensile and shear strains simultaneously, by a constant given equivalent strain amplitude, 1\%. In the computations the model parameters have been determined from the isothermal analysis, that is, the thermo-mechanical coupling is not considered in the constitutive modeling explicitly.

In Fig. \ref{Fig:TMF_IP} the hysteresis loops of the stabilized cycles are shown, in which the in-phase (IP), the out-of-phase (OP) and 90$^\circ$-phase temperature loads are considered, respectively.
The IP-phase loops show similar features as those of the isothermal loading.
The stress-strain curve reveals cyclic softening, as observed in isothermal tests.Interesting is the valley stress of the IP test seems higher than the computational prediction, which may be related to the thermo-mechanical effects. However, the effect seems not significant.

As soon as the temperature does not change proportionally to the mechanical load, the stress-strain loops become asymmetric, as shown in Fig. \ref{Fig:TMF_IP}(a).
%The stress-strain loops for the stabilized cycles, i.e. $N_f$/2th cycle, in Fig. \ref{Fig:TMF_IP}(a), \ref{Fig:TMF_OP}(a) and \ref{Fig:TMF_90}(a) confirm that the constitutive model can predict the thermo-mechanical behavior of the material properly.
Furthermore, Fig. \ref{Fig:TMF_IP}(b) shows that the peak and valley stress of the computational prediction in comparing with the experiments.
The peak stresses of all temperature loading phases, i.e. IP, OP and 90$^\circ$-phase, are confirm a good agreement between the experiments and computations.
However, for all valley stresses, the experimental results are observed approximately 50 MPa lower than the predictions.
The reason is that the yield surface is assumed to be circular in the present constitutive model.
As shown by Gil \cite{Gil1998}, the yield surface of Inconel 718 is shift to the compression direction as temperature increase from room temperature to 650$^{\circ}$C.
Because of the small deviations in the valley stresses, the yield surface is not further modified in the present work.

%The agreement between the experiments and the computations is obvious.

%\begin{figure*}[!htp]
%\centering{\includegraphics[width=8.5cm]{Chapter6Figs/7018_TMF_+-1_IP_10th.pdf}}
%\centering{\includegraphics[width=8.5cm]{Chapter6Figs/IN718_TMF_Axial+-1_PV_Exp_vs_Sim_IP.pdf}}
%\centering{\includegraphics[width=8.5cm]{Chapter6Figs/7017_TMF_+-1_OP_10th.pdf}}
%\centering{\includegraphics[width=8.5cm]{Chapter6Figs/IN718_TMF_Axial+-1_PV_Exp_vs_Sim_OP.pdf}}
%\centering{\includegraphics[width=8.5cm]{Chapter6Figs/7025_TMF_+-1_90_10th.pdf}}
%\centering{\includegraphics[width=8.5cm]{Chapter6Figs/IN718_TMF_Axial+-1_PV_Exp_vs_Sim_90.pdf}}
%\caption{Experimental and computational results under the multi-axial thermo-mechanical loading conditions with temperature between 300 and 650$^{\circ}$C. (a) The hysteresis loop in the in-phase loading. (b) Peak-valley stresses as function of the loading cycles in the in-phase loading. (c) The stress-strain hysteresis loop in the out-of-phase loading. (d) Peak-valley stress as a function of loading cycles in the out-of-phase loading. (e) The  stress-strain hysteresis loop with a phase angle of 90$^\circ$ . (f) Peak-valley stress as a function of loading cycles with a phase angle of 90$^\circ$ . \marked{(I would suggest to combine loops into one figure and peak-valleys into one figure. These 6 figures do not give so different informations. They are too similar to give explanations.)}}
%\label{Fig:TMF_IP}
%\end{figure*}

%\begin{figure*}[!htp]
%\centering{\includegraphics[width=8.5cm]{Chapter6Figs/7017_TMF_+-1_OP_10th.pdf}}
%\centering{\includegraphics[width=8.5cm]{Chapter6Figs/IN718_TMF_Axial+-1_PV_Exp_vs_Sim_OP.pdf}}
%\caption{Experimental and computational results under the out-of-phase multi-axial thermo-mechanical loading conditions with temperature 300-650$^{\circ}$C. (a) The stabilized stress-strain hysteresis loop. (b) Peak-valley stress as a function of loading cycles.}
%\label{Fig:TMF_OP}
%\end{figure*}

%\begin{figure*}[!htp]
%\centering{\includegraphics[width=8.5cm]{Chapter6Figs/7025_TMF_+-1_90_10th.pdf}}
%\centering{\includegraphics[width=8.5cm]{Chapter6Figs/IN718_TMF_Axial+-1_PV_Exp_vs_Sim_90.pdf}}
%\caption{Experimental and computational results under the multi-axial thermo-mechanical loading conditions with a phase angle of 90$^\circ$ and temperature of 300-650$^{\circ}$C. (a) The stabilized stress-strain hysteresis loop. (b) Peak-valley stress as a function of loading cycles.}
%\label{Fig:TMF_90}
%\end{figure*}


\begin{figure*}[!htp]
\centering{\includegraphics[width=8.5cm]{Chapter6Figs/IN718_TMF_IP_Prop45+-1_1st.pdf}}
\centering{\includegraphics[width=8.5cm]{Chapter6Figs/IN718_TMF_IP_Prop45+-1_50th.pdf}}
\centering{\includegraphics[width=8.5cm]{Chapter6Figs/IN718_TMF_IP_Prop45+-1_Exp_vs_Sim_PV_Axial.pdf}}
\centering{\includegraphics[width=8.5cm]{Chapter6Figs/IN718_TMF_IP_Prop45+-1_Exp_vs_Sim_PV_Torsional.pdf}}
\caption{Experimental and computational results under the proportional in-phase thermo-mechanical loading conditions with temperature of 300-650$^{\circ}$C. (a) The first loading cycle. (b) The half life loading cycle. (c) Axial peak-valley stresses as function of the loading cycles. (d) Torsional peak-valley stresses as function of the loading cycles.}
\label{Fig:TMF_Prop45}
\end{figure*}


\begin{figure*}
\centering{\includegraphics[width=8.5cm]{Chapter6Figs/IN718_TMF_7037_Diamond+-1_1st.pdf}}
\centering{\includegraphics[width=8.5cm]{Chapter6Figs/IN718_TMF_7037_Diamond+-1_20th.pdf}}
\centering{\includegraphics[width=8.5cm]{Chapter6Figs/IN718_TMF_IP_Diamond+-1_Exp_vs_Sim_PV_Axial.pdf}}
\centering{\includegraphics[width=8.5cm]{Chapter6Figs/IN718_TMF_IP_Diamond+-1_Exp_vs_Sim_PV_Torsional.pdf}}
\caption{Experimental and computational results under the diamond strain path in-phase thermo-mechanical loading conditions with temperature of 300-650$^{\circ}$C. (a) The first loading cycle. (b) The half life loading cycle. (c) Axial peak-valley stresses as function of the loading cycles. (d) Torsional peak-valley stresses as function of the loading cycles.}
\label{Fig:TMF_Diamond}
\end{figure*}

\subsection{Multi-axial thermo-mechanical fatigue tests}
Multi-axial thermo-mechanical material behavior have not been studied systematically in literature. Effects of multi-axial loads and temperature are not discussed. In the present work both proportional loading path (Fig. \ref{Fig:LoadPath}(b)) and non-proportional diamond loading path (Fig. \ref{Fig:LoadPath}(c)) are investgated. Experimental results are shown in Figs. \ref{Fig:TMF_Prop45} and \ref{Fig:TMF_Diamond}, together with computational predictions.

In both tests, the temperature ranges between 300 and 650$^{\circ}$C, the equivalent mechanical strain amplitude $\varepsilon_{eq,a}=1\%$, the phase angle between axial strain and temperature is zero degree, the in-phase loading (IP).
The torsional and axial stress curve reveals cyclic softening for all tested multi-axial loading cases.
Computations were performed with the non-proportional hardening factor $Y_{sat}=0$MPa and $Y_{sat}=200$MPa, respectively, to demonstrate the influence of the non-proportional correction to the computational results.

As shown in Fig. \ref{Fig:TMF_Prop45}, the proportional loading case provides good agreement between the computation and experiment in peak value distribution, while the valley show a similar deviation as observed in isothermal loading. The shifting in the shear stress-normal stress loop is caused by the under-estimate of the compressive normal stress. Generally, computations agree with the experiment reasonably. Additionally, the non-proportional hardening factor do not effect the computational results, as expected.

Under the non-proportional loading condition, the diamond loading path shows significant effects of the non-proportional hardening factor  in the constitutive model, as shown in Fig. \ref{Fig:TMF_Diamond}. The stress loops of the first loading cycles and the stabilized cycles are illustrated in Figs. \ref{Fig:TMF_Diamond}(a) and (b). In both first loading cycle and stabilized cycle the non-proportional hardening factor improves a much better computational prediction in comparing with the conventional model. Additional hardening due to the non-proportional loading can be observed and effects of the non-proportional hardening in the model are significant.

The axial and torsional peak-valley stresses are illustrated in  Figs. \ref{Fig:TMF_Diamond}(c) and (d).
In both tests one observes slight deviations between computational predictions and experiments after  loading direction changes. The stress amplitude at the low temperature seems to be under-estimated in comparing with the higher temperature point. However, it confirm that the constitutive model can predict the thermo-mechanical behavior of the material properly.

%\begin{figure*}
%  \begin{minipage}[t]{0.5\linewidth}
%    \centering
%    \includegraphics[width=3.5in]{Chapter6Figs/IN718_TMF_IP_Prop45+-1_1st.pdf}
%    \centerline{(a) The first loading cycle.}
%    \label{Fig:IN718_TMF_IP_Prop45+-1_1st}
%  \end{minipage}
%  \begin{minipage}[t]{0.5\linewidth}
%  \nonumber
%    \centering
%    \includegraphics[width=3.5in]{Chapter6Figs/IN718_TMF_IP_Prop45+-1_50th.pdf}
%    \centerline{(b) The 50th loading cycle.}
%    \label{Fig:IN718_TMF_IP_Prop45+-1_50th}
%  \end{minipage}%
%  \label{Fig:TMF_90}
%  \caption{Experimental and computational results under the proportional in-phase thermo-mechanical loading conditions with temperature of 300-650$^{\circ}$C.}
%\end{figure*}

%\begin{figure*}
%  \begin{minipage}[t]{0.5\linewidth}
%    \centering
%    \includegraphics[width=3.5in]{Chapter6Figs/IN718_TMF_7037_Diamond+-1_1st.pdf}
%    \centerline{(a) The first loading cycle.}
%    \label{Fig:IN718_TMF_7037_Diamond+-1_1st}
%  \end{minipage}
%  \begin{minipage}[t]{0.5\linewidth}
%  \nonumber
%    \centering
%    \includegraphics[width=3.5in]{Chapter6Figs/IN718_TMF_7037_Diamond+-1_20th.pdf}
%    \centerline{(b) The 50th loading cycle.}
%    \label{Fig:IN718_TMF_7037_Diamond+-1_20th}
%  \end{minipage}%
%  \label{Fig:TMF_90}
%  \caption{Experimental and computational results under the diamond strain path in-phase thermo-mechanical loading conditions with temperature of 300-650$^{\circ}$C.}
%\end{figure*}

\section{Conclusions}
In the present work the multi-axial thermo-mechanical evolution of the nickel-based superalloy Inconel 718 has been investigated experimentally and computationally.
Based on detailed experiments, it is suggested that the kinematic hardening is caused by the cyclic loading and the isotropic hardening is caused by the non-proportional loading path.
Based on the frame of the Ohno-Wang's model, a new constitutive model has been suggested for multi-axial thermo-mechanical cyclic plasticity.
The proposed constitutive model considers most different strain paths and different thermo-mechanical loading phases.
The implicit computational integration algorithm for the constitutive model has been developed and implemented into the general purpose commercial finite element code ABAQUS.
From the present experimental and computational investigation the following conclusions can be drawn:
\begin{itemize}

\item {Complex variations in the peak and valley stresses can be described by the kinematic hardening model properly. The numerical predictions presents good agreement with the experiments in varying temperatures.}

\item {Computations confirm the significance of non-proportional hardening under multi-axial loading conditions and agree with experiments under different loading conditions. Generally, the prediction of the constitutive model is reasonable under both proportional and non-proportional loadings.}

\item {The temperature-dependent material parameters are determined under isothermal conditions. The results reveal the constitutive model can approach thermo-mechanical behavior reasonably under both axial and multi-axial thermo-mechanical loadings.}

\item{The complex loading path may induce additional strain hardening in the compressive normal stress, which cannot be caught by the present constitutive model. To quantify this effect, more detailed experiments are necessary.}

\end{itemize}

%\section*{Acknowledgement:}

\section*{Acknowledgement}
The present work is financed by the China Natural Science Foundation under the contract number 11572169.

\section*{References}


\bibliographystyle{unsrt}            % bibliography style
\bibliography{bibliography}          % personal bibliography file



\end{document}


